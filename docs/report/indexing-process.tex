\section{Indexing Process}
To index the documents a custom Lucene Analyzer has been adopted.
The implemented custom Analyzer is based on the \textit{Lucene Classic Analyzer} in addition of some \textit{TokenFilter}.

\begin{enumerate}
    \item \textbf{Addresses filter}: discards URLs and email addresses. Tweets contain usually shortened urls which doesn not carry any kind of information.
    \item \textbf{Classic Tokenizer}: produces tokens using spaces and 
    punctuation as splitting point and discards them.
    Unlike the Standard Tokenizer, the Classic Tokenizer preserves URLs, email addresses and strings with alphabetic characters and numbers separated with hyphens. Special characters such as \verb|( ) [ ] &| and so on are considered splitting points (e.g. \quotes{AX-02}, \quotes{21:35}, \quotes{larry@google.com}, \quotes{www.duckduckgo.com} are preserved and \quotes{document-based} is splitted discarding hyphen).
    \item \textbf{Lower case transformation}.
    \item \textbf{Stop word removal}.
    \item \textbf{Porter stemming}.
\end{enumerate}

The structure of the Lucene Document that represent a Tweet is composed of several Fields, each of which is used to either perform queries on it or to compute a custom score for the matching phase. 
Here is reported the list of the employed Fields with the respective use.

\begin{itemize}
    \item \verb|TWEET_ID|: the Id of the tweet which is then used to retrieve it for display
    \item \verb|DATE|: creation date, used both to perform range queries and to chronologically order the results during the first matching phase described in \ref{subsec:rescoring-mechanism}.
    \item \verb|TEXT|: the textual content of the tweet on which the query is made.
    \item \verb|HASHTAG|: the list of hashtags used in the tweet used by the Search Engine jointly with \verb|TEXT|
    \item \verb|IS_QUOTE|, \verb|IS_RETWEET|, \verb|HAS_URLS|, \verb|RETWEET_COUNT|, \verb|FAVORITE_COUNT|, \verb|USER_FOLLOWERS|, \verb|USER_FOLLOWING|: fields used to re-rank results based on the nature of social platform (Twitter).
\end{itemize}