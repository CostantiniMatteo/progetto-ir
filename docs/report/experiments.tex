\section{Experiments}

The personalization has been tested searching for a generic word, such the word \quotes{car}, both in the non-personalized mode and in the personalized mode (with topic \quotes{Cars} and user profile with an interest for brands of car such \quotes{BMW} and \quotes{Tesla}).
The expected and empirically confirmed behaviour is a moderate modification of the results: new tweets about the user interests appear and the generic ones go down.
In case of chronological search the Search Engine preserve the timeline and do not let many new tweets emerge.
Using the same user as the previous example, which is interested in \textit{BMW} and \textit{Tesla}, and choosing again the topic \textit{Cars} to perform a personalized search, if the query is based on words like \textit{turtle} which is not related in any sense with the chosen topic, the personalization does not affects the results.
An analogous experiment is searching for an ambiguous word such as \quotes{stock} and looking how a personalization with topic \quotes{finance} and user profile with interest in some corporations and market trends provide enough contextual information to disambiguate results. 
